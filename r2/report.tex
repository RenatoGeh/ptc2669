\documentclass{amsart}

\usepackage[brazilian]{babel}
\usepackage[utf8]{inputenc}
\usepackage{graphicx}
\usepackage{mathtools}
\usepackage{amsthm}
\usepackage{thmtools,thm-restate}
\usepackage{amsfonts}
\usepackage{hyperref}
\usepackage[singlelinecheck=false]{caption}
\usepackage[backend=biber,url=true,doi=true,eprint=false,style=alphabetic]{biblatex}
\usepackage{enumitem}
\usepackage[justification=centering]{caption}
\usepackage{indentfirst}
\usepackage{algorithm}
\usepackage{algpseudocode}
\usepackage{listings}

\addbibresource{references.bib}

\makeatletter
\def\subsection{\@startsection{subsection}{3}%
  \z@{.5\linespacing\@plus.7\linespacing}{.1\linespacing}%
  {\normalfont\itshape}}
\makeatother

\DeclareMathOperator*{\argmin}{arg\,min}
\DeclareMathOperator*{\argmax}{arg\,max}

\newcommand\defeq{\mathrel{\overset{\makebox[0pt]{\mbox{\normalfont\tiny\sffamily def}}}{=}}}

\floatname{algorithm}{Algoritmo}
\algrenewcommand\algorithmicrequire{\textbf{Input}}
\algrenewcommand\algorithmicensure{\textbf{Output}}

\captionsetup[table]{labelsep=space}

\theoremstyle{plain}

\newcounter{dummy-def}\numberwithin{dummy-def}{section}
\newtheorem{definition}[dummy-def]{Definição}
\newcounter{dummy-thm}\numberwithin{dummy-thm}{section}
\newtheorem{theorem}[dummy-thm]{Teorema}
\newcounter{dummy-prop}\numberwithin{dummy-prop}{section}
\newtheorem{proposition}[dummy-prop]{Proposição}
\newcounter{dummy-ex}\numberwithin{dummy-ex}{section}
\newtheorem{exercise}[dummy-ex]{Exercício}
\newcounter{dummy-eg}\numberwithin{dummy-eg}{section}
\newtheorem{example}[dummy-eg]{Exemplo}

\numberwithin{equation}{section}

\newcommand{\set}[1]{\mathbf{#1}}
\newcommand{\pr}{\mathbb{P}}
\renewcommand{\implies}{\Rightarrow}

\newcommand{\bigo}{\mathcal{O}}

\setlength{\parskip}{1em}

\lstset{frameround=fttt,
  language=[5.3]Lua,
	numbers=left,
	breaklines=true,
	keywordstyle=\bfseries,
	basicstyle=\ttfamily,
}

\newcommand{\code}[1]{\lstinline[mathescape=true]{#1}}
\newcommand{\mcode}[1]{\lstinline[mathescape]!#1!}


\title{%
  \noindent\rule{10cm}{0.8pt}\\
  Estudo sobre Sum-Product Networks e Aprendizagem Profunda\\[1ex]
  \scriptsize\mdseries
  PTC2669 - Introdução a Inteligência Computacional\\
  Instituto de Matemática e Estatística - USP\\%
  \tiny~\\
  Relatório 2\\%
  \noindent\rule{10cm}{0.8pt}
}
\xdef\shorttitle{Sum-Product Networks e Deep Learning (Relatório 2) - Renato Geh}
%\title[]{Estudo sobre Sum-Product Networks e Aprendizagem Profunda}
\author[]{Renato Lui Geh\\NUSP\@: 8536030}

\begin{document}

\begin{abstract}
  Sum-Product Networks (SPNs) são modelos probablísticos baseados em grafo que representam uma
  distribuição de probabilidade por meio de uma função multilinear dos parâmetros de uma Rede
  Bayesiana. Neste relatório, definiremos uma SPN recursivamente e mostraremos algumas propriedades
  interessantes da rede. Em seguida mostraremos como executar aprendizagem estrutural de uma SPN
  de forma que consigamos criar uma estrutura profunda a partir dos dados observados.
  \vspace*{-3.5em}
\end{abstract}

\maketitle

\section{Definições}

No relatório anterior, definimos uma Sum-Product Network (SPN) como um DAG com nós soma, produto
e indicadores. Vamos transcrever a definição abaixo.

\begin{definition}\label{pd-def}
  Uma SPN $S$ é um DAG com três tipos de nós: soma, produto e indicadores. Todo nó indicador é uma
  folha. Todo nó soma tem pais produto, e todo nó produto tem pais soma. Toda aresta com destino a
  um nó soma tem uma aresta com um peso associado. O valor de um nó soma $i$ é $\sum_{j\in Ch(i)}
  w_{ij}v_j$ e o valor de um nó produto $i$ é $\prod_{j\in Ch(i)}v_j$, onde $Ch(i)$ é o conjunto
  de filhos de $i$, $v_i$ é o valor do nó $i$ e $w_{ij}$ é o peso associado a aresta $i\to j$. Uma
  SPN representa um \emph{network polynomial} de uma distribuição de probabilidade, e os
  indicadores da função são as folhas da SPN\@. O valor de uma SPN é o valor do nó raíz.
\end{definition}

Considere um nó arbitrário $i$ de uma SPN, então a SPN representada por $i$ é uma SPN e é denotada
por $S_i(\cdot)$.

\begin{proposition}
  Seja um nó arbitrário $i$ de uma SPN $S$, então $S_i$ é uma sub-SPN que tem nó raíz em $i$.
\end{proposition}

\begin{proof}
  Considere o caso base em que $i$ é um nó indicador. Um nó indicador é uma distribuição de
  probabilidade monovariável. Portanto $i$ é uma distribuição de probabilidade e pode ser
  representada por uma SPN, que no caso possue apenas um nó.

  Se $i$ é um nó soma, então o valor de $i$ é $v_i=\sum_{j\in Ch(i)} w_{ij}v_j$. A soma de várias
  distribuições de probabilidade é uma distribuição de probabilidade. Portanto um nó soma é
  representável por uma SPN\@.

  Caso $i$ seja um nó produto, então o valor de $i$ é $v_i=\sum_{j\in Ch(i)} v_j$. A multiplicação
  de distribuições de probabilidade é bem definida e é uma distribuição de probabilidade. Um nó
  produto é uma SPN\@.
\end{proof}

O escopo de uma SPN $S$ é o conjunto união de todos os escopos de seus filhos. Denotaremos por
$Sc(S_i)$ o escopo da SPN $S_i$.

\section{Propriedades}

Vamos abordar duas propriedades que são essenciais para definirmos uma SPN que computa de forma
eficiente probabilidades marginais.

\begin{definition}[Completude]
  Uma SPN $S$ é completa se e somente se, para todo nó soma $i$, o escopo de $i$ é igual par-a-par
  $Sc(S_i)=Sc(S_j)$ ao escopo de cada filho $j\in Ch(i)$.
\end{definition}

Em outras palavras, se tivermos duas distribuições de probabilidade $p_1(\set{X})$ e
$p_2(\set{X})$, onde seus escopos são $\set{X}$, uma SPN $S$ que representa a soma de $p_1$ e $p_2$
é completa sse $Sc(S)=Sc(p_1)=Sc(p_2)=\set{X}$.

\begin{definition}[Consistência]
  Uma SPN $S$ é consistente se e somente se, para todo nó produto $i$, nenhum filho de $i$ tem
  valor diferente dos outros filhos.
\end{definition}

A propriedade de consistência garante que não haja contradição entre as variáveis indicadoras, ou
seja, o valor de uma variável indicadora não pode ter dois valores simultâneos.

\begin{definition}[Validade]
  Uma SPN $S$ é válida se $S$ é consistente e completa.
\end{definition}

Note que a volta não é necessariamente verdade. De fato existem SPNs válidas que não são
consistentes nem completas. No entanto, se uma SPN é completa e consistente, então todas as suas
sub-SPNs são válidas. A validade de uma SPN garante que inferência seja computada em tempo
eficiente e que seu resultado seja exato. Uma SPN que não seja válida computa uma probabilidade
aproximada.

\begin{definition}[Decomponibilidade]
  Uma SPN $S$ é decomponível se e somente se, para cada par $c_1, c_2 \in Ch(i)$ para qualquer $i$
  nó produto em $S$, $Sc(c_1)\cap Sc(c_2)=\emptyset$.
\end{definition}

Ou seja, todo filho de um nó produto tem escopos diferentes entre si. Uma SPN decomponível é uma
SPN consistente, no entanto consistência não implica em decomponibilidade.

\section{Uma definição alternativa}

Em~\cite{gens-domingos}, foi proposto uma nova definição de SPNs. Esta definição, é mais forte que
a definição dada em~\cite{poon-domingos} (cuja Definição~\ref{pd-def} se baseia), já que, como
veremos, assume já que a distribuição possa ser representada por uma SPN completa e decomponível.
Esta definição, de mais ``alto nível'', nos permite mudar o foco de nossos estudos,
distanciando-nos da área de Representação e voltando nossos interesses ao tópico de aprendizado e
inferência.

\begin{definition}\label{gd-def}
  Uma SPN tem uma definição recursiva. Definimos que uma SPN $S_i$ pode ser apenas:
  \begin{enumerate}[label=(\roman*)]
    \item\label{gd-ref-1} Uma distribuição monovariável $p(\set{X})$ ou;
    \item\label{gd-ref-2} Um nó soma tal que $S_i=\sum_{j\in Ch(i)} w_{ij}v_j$ onde para cada filho
      $j,k\in Ch(i)$, $Sc(S_j)=Sc(S_k)$ ou;
    \item\label{gd-ref-3} Um nó produto tal que $S_i=\prod_{j\in Ch(i)} v_j$ onde para cada filho
      $j,k\in Ch(i)$, $Sc(S_j)\cap Sc(S_k)=\emptyset$.
  \end{enumerate}
\end{definition}

Note que em~\ref{gd-ref-2}, estamos assumindo completude em nós somas. Da mesma forma,
regra~\ref{gd-ref-3} impõe decomponibilidade a nós produtos. Com relação a~\ref{gd-ref-1},
restringimos apenas que $p(\set{X})$ seja tratável, ou seja, supomos que exista uma Rede Bayesiana
ótima que compute $p(\set{X})$ de forma tratável (sub-exponencial). Por exemplo, uma distribuição
gaussiana.

\section{Classificação por Naive Bayes}

Um modelo Naive Bayes é uma subclasse de Redes Bayesianas onde cada nó atributo tem um mesmo pai.
Neste modelo, temos um nó classe que representa a classificação que desejamos encontrar, enquanto
que cada nó atributo é uma característica que influencia no resultado da classificação. Como cada
nó atributo possue uma única aresta com origem no nó classe, cada atributo é d-separado par-a-par.
Esta suposição, apesar de restritiva, obtém resultados bons na prática. Esse relaxamento de
dependência entre os atributos permite aprendizado rápido, simples e fácil.

Pelo Teorema da Fatorização, a distribuição de probabilidade conjunta de um modelo Naive Bayes com
$n$ atributos é computada como $\Pr(C,A_1,\ldots,A_n)=\Pr(C)\prod_{i=1}^n \Pr(A_i|C)$, onde $C$ é
a variável classe e $\set{A}=\{A_1,\ldots,A_n\}$ é o conjunto de atributos. Para acharmos a
explicação mais provável para uma certa instanciação $\set{a}=\{a_1,\ldots,a_n\}$, maximizamos a
instanciação da classe $\argmax_c \Pr(C=c|A_1=a_1,\ldots,A_n=a_n)$. A probabilidade posteriori
pode ser encontrada marginalizando sob $C$ e $\set{A}$: $\Pr(C=c|A_1=a_1,\ldots,A_n=a_n)=\Pr(C=c,
A_1=a_1,\ldots,A_n=a_n)/\Pr(A_1=a_1,\ldots,A_n=a_n)$.

Aprendizado em Naive Bayes pode ser feita por uma estimação da máxima verossimilhança (MLE). Para a
variável classe, $\Pr(C=c)=\frac{N[C=c]}{N}$ onde $N[X=x]$ é a contagem de cada dado em que houve
a ocorrência da variável $X$ instanciada a $x$ e $N$ é o número de contagens totais. Para cada
atributo $i$, o MLE é dado por $\Pr(A_i=a_i|C=c)=\frac{N[A_i=a_i,C=c]}{N[C=c]}$.

Quando há poucos dados, a variância das probabilidades pode tornar-se demasiadamente grande. Para
suavizarmos os resultados, podemos aplicar um fator de suavização somando alguma constante aos
termos de cada atributo.

\section{Uma algoritmo de aprendizado estrutural}

Nesta seção descreveremos simplificadamente um método de se aprender a estrutura de uma SPN a
partir dos dados. Este método pode ser lido em detalhes em~\cite{gens-domingos}.

Para este algoritmo, supomos que estamos construindo um classificador baseado em Naive Bayes, ou
seja, as características para classificação são independentes entre si.

\begin{algorithm}
  \caption{LearnSPN}\label{learn-alg}
  \begin{algorithmic}[1]
    \Require~Conjunto $\set{X}$ de variáveis, conjunto $\set{I}$ de instâncias
    \Ensure~Uma SPN resultante do aprendizado estrutural
    \If{$|\set{X}|=1$}
      \State~Retorna uma distribuição monovariável de $\set{X}$
    \EndIf
    \State~Tente dividir as variáveis $\set{X}$ em duas partições $\set{X}_1$ e $\set{X}_2$ onde
      $\set{X}_1$ é (aproximadamente) independente de $\set{X}_2$
    \If{dá para dividir}
      \State~\textbf{return} $\prod_{i=1}^2$ LearnSPN($\set{X}_i$, $\set{I}$)
    \Else
      \State~Divida as instâncias $\set{I}$ em partições $\set{I}_1$ e $\set{I}_2$ tal que
        $\set{I}_1$ e $\set{I}_2$ sejam o mais similares possíveis.
      \State~\textbf{return} $\sum_{i=1}^2 \frac{|\set{I}_i|}{|\set{I}|}$ LearnSPN($\set{X}$,
        $\set{I}_i$)
    \EndIf
  \end{algorithmic}
\end{algorithm}

No Algoritmo~\ref{learn-alg}, descrevemos como aprender a estrutura de uma SPN\@. O algoritmo
procura buscar independências entre as variáveis dos dados. Se houver uma possível partição nas
variáveis, ou seja, existe uma independência entre uma partição do conjunto de variáveis e o
complemento dela, podemos então criar um nó produto cujos filhos são as sub-SPNs cujos escopos são
as partições mas que possuem as mesmas instanciações. Note que obedecemos a propriedade de
decomponibilidade, já que como cada atributo é independente par-a-par, teremos escopos diferentes
para cada filho de um nó produto. Para acharmos independências, podemos recorrer a um teste
estatístico sobre os dados. Queremos minimizar os erros do tipo II (falso negativo), já que
preferimos minimizar casos em o teste diz que as partições são independentes quando não são.
Para isso podemos controlar os erros do tipo I (falso positivo), onde o teste diz que não são
independentes mas na realidade os eventos são de fato independentes.

Se não foi possível criar um nó produto, então resta o caso em que podemos criar um nó soma. Os nós
somas são divididos de tal forma que as instâncias sejam mais similares possíveis. Isso possibilita
que a SPN não fique desbalanceada, e que as camadas de nós sejam o mais ``profundas'' possíveis.

Quando chegamos ao caso base onde temos apenas uma variável em $\set{X}$, retornamos uma
distribuição monovariável como nó folha.

Para podermos visualizar mais graficamente, considere uma matriz de dados onde as linhas
representam as instanciações e as colunas as variáveis. Acharmos duas partições $\set{X}_1$,
$\set{X}_2$ é graficamente equivalente a dividir a matriz em duas submatrizes verticalmente.
Podemos então computar, recursivamente, uma sub-SPN usando uma das sub-matrizes e a outra com o seu
complementar. Analogamente, a divisão de instâncias semelhantes $\set{I}_1$, $\set{I}_2$ equivale
a uma divisão horizontal da matriz. Quando temos uma submatriz de tamanho $1\times n$, temos uma
distribuição monovariável, que no caso equivale a um vetor das valorações de uma única variável.

%--------------------------------------------------------------------------------------------------

\newpage
\appendix

\newpage

\printbibliography[]

\end{document}
